\newpage
\section{Fazit und Ausblick}

Die Gruppe wurde während des ganzen Praktikums vor immer neue Herausforderungen gestellt, die es gemeinsam zu Lösen galt. Die Einarbeitung in ein neues, unbekanntes Thema, die Programmiersprache und das Zusammenarbeiten im Team über eine größere Distanz bzw. nur Online war zu Beginn für alle neu. Durch eine offene Kommunikation hat jeder etwas beitragen können, jedes Gruppenmitglied konnte seine Erfahrungen aus dem eigenen Bereich einbringen, so dass die Gruppe erfolgreich an den Turnieren teilgenommen hat. \newline

Während der ganzen Zeit traten verschiedene Probleme auf, welche kontinuierlich diskutiert, bewertet und bearbeitet wurden. Die wöchentlichen Treffen haben zu einem guten Austausch geführt und jeden einzelnen in der Entwicklung weitergebracht. Aufgrund einiger erfolgloser Turniere musste das Team lernen, auch mit einer Niederlage oder Enttäuschung umzugehen. \newline

Die \Agents haben noch Potential, welches wir aus Zeitgründen nicht ausschöpfen konnten. Z.B. haben wir für serverseitige \textit{Events} noch keine Reaktion implementiert. Das Bilden von Kleingruppen mit den ersten beiden Agenten, die sich über den Weg laufen, ist ausbaufähig. Die \Agents haben zu oft aufeinander gewartet. Hierfür wäre eine dynamische Zusammenarbeit der \Agents zielführender  gewesen. Die anderen Agenten wurden primär als Fremdkörper interpretiert, denen ausgewichen werden konnte, jedoch wurde keine aktive Interaktion mit diesen angestrebt. Desweiteren könnte die Kommunikation innerhalb der Gruppe für bessere Koordination weiter ausgebaut werden. 

