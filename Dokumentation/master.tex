% This is samplepaper.tex, a sample chapter demonstrating the
% LLNCS macro package for Springer Computer Science proceedings;
% Version 2.21 of 2022/01/12
%
\documentclass[runningheads]{llncs}
\pagenumbering {gobble}
%
\usepackage[T1]{fontenc}
\usepackage{hyperref}
% T1 fonts will be used to generate the final print and online PDFs,
% so please use T1 fonts in your manuscript whenever possible.
% Other font encondings may result in incorrect characters.
%
\usepackage{graphicx}
% Used for displaying a sample figure. If possible, figure files should
% be included in EPS format.
%
% If you use the hyperref package, please uncomment the following two lines
% to display URLs in blue roman font according to Springer's eBook style:
%\usepackage{color}
%\renewcommand\UrlFont{\color{blue}\rmfamily}
%
\usepackage[ngerman]{babel}

% Commands for specific words to show in italics. \Agent shows the word "Agent" in italics. 
\newcommand{\Agent}{\textit{Agent} }
\newcommand{\Agents}{\textit{Agents} }
\newcommand{\clear}{\textit{clear} }
\newcommand{\Dispenser}{\textit{Dispenser} }
\newcommand{\GoalZone}{\textit{Goal Zone} }
\newcommand{\GoalZones}{\textit{Goal Zones} }
\newcommand{\move}{\textit{move} }
\newcommand{\NextGroup}{\textit{NextGroup} }
\newcommand{\NextMap}{\textit{NextMap} }
\newcommand{\NextMapTile}{\textit{NextMapTile} }
\newcommand{\NextMapTiles}{\textit{NextMapTiles} }
\newcommand{\Obstacle}{\textit{Obstacle} }
\newcommand{\Obstacles}{\textit{Obstacles} }
\newcommand{\Percept}{\textit{Percept} }
\newcommand{\Percepts}{\textit{Percepts} }
\newcommand{\RoleZone}{\textit{Role Zone} }
\newcommand{\RoleZones}{\textit{Role Zones} }
\newcommand{\Step}{\textit{Step} }
\newcommand{\Steps}{\textit{Steps} }
\newcommand{\Thing}{\textit{Thing} }
\newcommand{\Things}{\textit{Things} }

\begin{document}
%
\title{Fachpraktikum Künstliche Intelligenz: Multiagentenprogrammierung}
%
%\titlerunning{Abbreviated paper title}
% If the paper title is too long for the running head, you can set
% an abbreviated paper title here
%
\author{Alexander Lorenz \and
Miriam Wolf \and
Sebastian Loder \and
Jan Steffen Jendrny}
%
\authorrunning{Lorenz A., Wolf M., Loder S., Jendrny S.}
% First names are abbreviated in the running head.
% If there are more than two authors, 'et al.' is used.
%
\institute{Fernuniversität Hagen, Universitätsstraße 47, 58097 Hagen
\url{https://www.fernuni-hagen.de/}}
%
\maketitle              % typeset the header of the contribution
%
%\begin{abstract}
%The abstract should briefly summarize the contents of the paper in
%150--250 words.
%
%\keywords{First keyword  \and Second keyword \and Another keyword.}
%\end{abstract}
%
%
%

%Einleitung
\section{Einleitung}

In diesem Kapitel wird die Auswahl der technischen Rahmenbedingungen, mit denen die Gruppe gearbeitet hat, genauer erläutert. Zudem erfolgt eine kurze Beschreibung der die Aufgaben, die von den Teammitgliedern bearbeitet wurden.

\subsection{Auswahl technischer Rahmenbedingungen}
Da die Schnittstelle der Programmierkenntnisse aller Gruppenmitglieder auf Java fiel und dies auch mit der MASSim Umgebung sowie der Vorlage des Agenten korrelierte, wurde in dieser Sprache entwickelt. Unsere Agentenvariante lief unter der Bezeichnung \textit{NextAgent}.  Alle Klassendateien beginnen mit dem Präfix \textit{Next-}, um die Zuordnung im Sourcecode zu erleichtern. Debugging erfolgte je nach Präferenz des Teilnehmers sowohl über Printausgabe in Echtzeit, als auch über Logfiles und Debugger. In kritischen Bereichen wurden Unittests umgesetzt. \\

Innerhalb des Quellcodes wurde auf Wunsch eines Teammitglieds eine spezielle Formatierung eingeführt. Dabei wurden die \textit{public} Methoden großgeschrieben, während \textit{private} Methoden weiterhin klein geschrieben wurden. Somit können öffentliche Methoden schneller erkannt und verwendet werden.

\subsection{Aufteilung Gruppenmitglieder - Arbeit}

Mit dem Ziel, den erarbeiteten Stand und weiteres Vorgehen zu besprechen, wurden wöchentliche Treffen abgehalten. Dieses konnten sehr konsequent umgesetzt werden. Initial wurde innerhalb der Gruppe versucht, in agiler Form mit Hilfe von \textit{issues} in GitHub zu arbeiten. Aufgrund der Komplexität des Themas und mangelnder Vorerfahrung der Teammitglieder, insbesondere im Kontext der Multiagentensysteme, haben wir mehr Zeit für die Einarbeitung des tieferen Verständnisses benötigt. 
\newpage
Aus diesem Grund wurden unser Konzept in vier Themenbereiche aufgeteilt, die wie folgt zugewiesen wurden:

\begin{itemize}
    \item \textbf{Herr Jendrny:} Festlegen der Aufgaben für den Agenten, basierend auf \textit{Tasks}, Normen und dem aktuellen Zustand der Welt.  
    \item \textbf{Frau Wolf:} Reaktion des Agenten auf die Umgebung und Wahl der auszuführenden \textit{Action}, mit Umsetzung der reaktiven Anteile.
    \item \textbf{Herr Lorenz:} Verarbeitung der \Percepts und Interaktion mit dem Server, optimale Wegfindung und Gruppenbildung, sowie Kommunikation.
    \item \textbf{Herr Loder:} Verarbeitung der Karte und das Zusammenführen der Karten für die Gruppe.
\end{itemize}

Zusätzlich gab es spontan koordinierte Treffen in kleinen Teams, um Schnittstellen zu diskutieren und Lösungen für Teilbereiche zu erarbeiten. 

\subsubsection{Alexander Lorenz} ~\\
Herr Lorenz setzte die Grundvariante des Agenten um, welche die Basis für weitere Entwicklung bildete. Der Schwerpunkt lag auf der sauberen Verarbeitung der Serverdaten, der Möglichkeit zwischen den Simulationen zu wechseln und der Fähigkeit, die ersten \textit{actions} an den Server zu senden. Dabei reagierte der Agent noch rein reaktiv und wählte die nächste Aktion basierend auf einer vorgegebenen Priorisierung. \newline

Im Verlauf des Fachpraktikums setzte Herr Lorenz die Logik der Gruppenbildung um, sowie eine Variante der Kommunikation innerhalb der Gruppe. Des Weiteren beschäftigte er sich mit den Möglichkeiten zur Optimierung der Wegfindungsalgorithmen und erweiterte die initiale A* Variante, um weitere Modifikationen, die in Kapitel \textit{\ref{kap:wegfindung}} genauer beschrieben werden. Als Schnittstelle zwischen Percepts, Karte und Entscheidungsfindung wurde viel Zeit in das Bugfixing, Fehlersuche, sowie die Interpretation des Agentenverhaltens investiert. \\

Bei Herrn Lorenz lag ein Fehler in der Konfiguration der IDE, so dass Github Commits nicht sauber zugeordnet waren. Dies wurde Anfang Juni bemerkt, und behoben.  

\subsubsection{Miriam Wolf} ~\\
Um einen Schritt nach vorn zu gehen oder einen Block zu zerstören, müssen die Agenten eine \textit{Action} an den Server schicken. Um einen optimalen Weg durch die Blöcke zu finden und sich zur Endzone durchzukämpfen, muss die nächstmögliche Aktion herausgefunden werden. Frau Wolf hat sich um die Möglichkeit gekümmert, welche Aktion die nächstmögliche ist. Sie hatte die Aufgabe, das Verhalten der Agenten in der lokalen Sicht zu bearbeiten. Die Erklärung der lokalen Sicht des Agenten wird später in Kapitel \ref{kap:lokaleSicht} genauer erläutert. \\

Weiter hatte sie die Aufgabe der Praktikumsorganisatorin inne. Hierfür musste sie die Treffen der Gruppenleiter koordinieren, die Turnierplanung übernehmen und auch die Turniere begleiten. Dazu gehörten unter anderem das Erstellen der Turnierkonfiguration oder das Starten des Servers selbst beim Turnier. Abschließend kam noch die Planung des Präsentationsnachmittags dazu und das Ausarbeiten des allgemeinen Dokumentationsteils.

\subsubsection{Sebastian Loder} ~\\
Kommt noch :)

\subsubsection{Steffen Jendrny} ~\\
Herr Jendrny hat sich um die Aufgabenplanung und -verteilung unter den Agenten gekümmert. Zu Beginn der Entwicklung hat jeder Agent selbstständig entschieden, welche Aufgabe ausgewählt wird und welche Teilaufgabe momentan erfüllt werden muss. Mit der Einführung von Agentengruppen wurde die Aufgabenauswahl auf die Gruppe ausgelagert und ein Agent hat lediglich entschieden, welche Teilaufgabe momentan erfüllt werden muss. Außerdem hat Herr Jendrny zusammen mit Frau Wolf die Abgabe von Aufgaben mit mehreren Blöcken implementiert. \\
Zusammen mit Frau Wolf war Herr Jendrny bei den Lead-Treffen dabei, um Gruppe 5 zu vertreten.

%% Softwarearchitektur
\section{Softwarearchitektur}

\subsection{Strategie}


\subsection{Erkundung der Karte} \label{erkundungDerKarte}

\subsubsection{Kreisförmige Karte} ~\\

Hier geht es darum, dass die Karte keinen Rand hat. \\
Kurz beschreiben, dass wir zu Beginn eine begrenzte Karte hatten und dann eine offene Karte

\subsection{Entscheidungsverhalten der Agenten}

\subsubsection{Rollen} ~\\
Die Rollen werden in der Serverkonfiguration vorgegeben. In den Turnieren standen fünf Rollen bereit: default, worker, constructor, explorer und digger. \\
Die Rollen haben verschiedene Aktionen, welche sie ausführen dürfen. Zu Beginn haben alle Agenten die \textit{default}-Rolle inne. Diese enthält alle Standardaktionen wie \textit{move, rotate, skip, adopt, detach und clear}. Um nach verschiedenen Dingen suchen zu können, benötigen die Agenten die Rolle \textit{explorer}. Mit diesen kann ein \textit{survey} nach einem Dispenser gemacht werden. Sobald die Agenten die Aufgabe bekommen einen Block abzuholen, müssen sie in die Rolle \textit{worker} wechseln. Damit können sie Blocke aufnehmen, sich mit anderen Agenten verbinden und auch die Aufgaben abgeben. 
Die Besonderheit ist, dass alle Rollen die Aktionen der default-Rolle erben. Die Standardaktionen besitzen somit alle anderen Rollen ebenfalls.

Die Rolle \textit{constructor} und \textit{digger} hat die Gruppe nicht weiter verfolgt, da diese für unsere Strategie nicht notwendig war. 

\subsubsection{Aufgaben} ~\\
NextTaskPlanner umgesetzt

Liste der Pläne: 
exploreMap,\\
goToDispenser,\\
goToGoalzone,\\
goToRolezone,\\
solveTask,\\
surveyDispenser,\\
surveyGoalZone,\\
surveyRoleZone,\\
surveyRandom,\\
connectToAgent,\\
discoverMapSize,\\
cleanMap,\\

\subsubsection{Wegfindung} \label{kap:wegfindung} ~\\
Random \newline
Spirale \newline
Manhattan \newline
A*

\subsubsection{Gruppenbildung} \label{kap:Gruppenbildung} ~\\

Gruppenbildung

\subsection{Globale und lokale Sicht}
Die Sicht der Agenten können in eine globale und eine lokalen Sicht unterschieden werden. Die globale Sicht besteht aus einer gespeicherten Karte pro Agent, die sich durch die Bewegungen des Agenten auf der Karte erweitert. Hier werden die verschiedenen Dinge wie Dispenser, Blöcke oder Zonen gespeichert. Sobald sich die Agenten in Gruppen finden, werden die Karten synchronisiert. Eine genauere Erklärung zu den Karten befindet sich im Kapitel \textit{\ref{erkundungDerKarte}}.\\

Die lokale Sicht des Agenten ist auf eine festgelegte Größe beschränkt. Die Sichtweite des Agenten wird in der Serverkonfiguration festgelegt. Bei einer Sichtweite von 5 sieht der Agent 5 Kästchen nach links, rechts, oben und unten, wie in Abbildung \ref{fig:agentensicht} dargestellt.
\begin{figure}
	\centering
	\includegraphics[width=150px]{bilder/agentensicht}
	\caption{Sicht des Agenten}
	\label{fig:agentensicht}
\end{figure}

Im ersten Schritt ermittelt der Agent einen möglichen Weg, wie er zu seinem Ziel gelangt. Hierfür wird der entsprechende Wegalgorithmus verwendet, welcher in Kapitel \textit{\ref{kap:wegfindung}]} genauer erläutert wird. Sobald der Weg ermittelt wurde und bevor ein Schritt des Weges beschritten wird, prüft der Agent auf Aktionen, welche vorher ausgeführt werden müssen. Diese Aktionen können einen Rollenwechseln beinhalten, ungenutzte Blöcke fallen lassen, einen Block vom Dispenser anfordern oder einen Block aufnehmen, eine Aufgabe in der Endzone abgeben oder sich mit einem anderen Agenten verbinden. Diese Aktionen sind Momentaufnahmen, die der Agent ausführen muss, bevor er zu einem neuen Ziel geht. Wenn keine dieser Aktionen passt, wird der Agent den Weg zu seinem Ziel gehen. Hier kommt nochmals eine Entscheidungsmöglichkeit für den Agenten in Frage. Ist der nächste Schritt frei und ich kann den Weg gehen, dann geht er diesen. Sollte aber beispielsweise ein Block in der Richtung sein, in die der Agent gehen möchte, so muss er diesen zunächst zerstören. Wenn ein Agent in der Richtung steht, so wird ein Weg um diesen Agenten herum erstellt. In Abbildung \ref{fig:agentensicht} wäre der Weg in Richtung Westen durch einen Block gehindert, sodass er diesen zunächst zerstören muss, bevor er in diese Richtung gehen kann. 

Zusammengefasst muss der Agent in jedem Schritt entscheiden, ob es eine Aktion gibt, die gerade notwendig ist, wie einen Block aufzunehmen oder ob der Schritt, den er gehen möchte, möglich ist.  

\subsection{Synchronisation und Kommunikation}
Die Kommunikation der Agenten funktioniert, sobald sie sich in einer Gruppe befinden. Die Synchronisation der Gruppen ist in Kapitel \textit{\ref{kap:Gruppenbildung}} genauer beschrieben. Für die Kommunikation wurde eine Schnittstelle entwickelt, welche die Nachricht, den Senderagenten und den Empfängeragenten in einer Nachrichtenbox bereithält. Stehen zwei Agenten um einen Dispenser und beide möchten einen Block anfordern, so wird der Agent zunächst prüfen, ob eine Nachricht für ihn vorliegt. Ist dies nicht der Fall, untersucht der Agent, ob andere Agenten in der nähe des Dispensers stehen. Wenn die Prüfung erfolgreich ist, so wird er eine Nachricht an den Agenten senden und dieser wird dann warten, bis der Dispenser frei ist. Er selbst stellt eine Anfrage an den Dispenser und nimmt den Block dann auf. 

%% Turniere
\section{Turniere}

\subsubsection{Turnier 1}
Das erste Turnier wurde nach einer ziemlich kurzen Entwicklungszeit abgehalten. Um die Erfahrung eines Turniers aber nicht zu verlieren, nahm das Team dennoch teil. Zu diesem Zeitpunkt waren die Agenten gerade in der Lage, die Karte zu erkunden. Der Fokus der ersten Entwicklungszeit lag auf der Erkundung der Karte, der Kommunikation mit dem Server und sich mit den Aufgaben und den Gegebenheiten des Turniers vertraut zu machen. Aufgaben konnten zu dem Zeitpunkt nicht abgegeben werden.\\
Das Ziel für das nächste Turnier war es, die Blöcke zu zerstören und zu den verschiedenen Zonen und Dispensern zu gehen. Eine bessere Wegfindung war ebenfalls notwendig.

%%Tasks mit Blöcken = 1 \newline
%Erster Task der liste \newline
%Agenten liefen mittels Random-Weg durch die Gegend. \newline
%Beim Entdecken vom Dispenser sind sie hingelaufen und haben sich einen Block genommen \newline
%Weiter random durch die Gegend bis zur Endzone \newline
%
%Probleme hier: Der Agent hat sich den Weg noch nicht freigehauen, somit hatten wir keine gute Methode zur Endzone zu kommen \newline
%Verbesserung: Bessere Wegfindung und wege freihauen

\subsubsection{Turnier 2}
Beim zweiten Turnier wurde die Kartierung verändert, warn allerdings immernoch simpel. Der Agent ist die Karte mit einem spiralförmigen Weg abgelaufen und hat somit verschiedene Dinge entdeckt. Sobald er einen Dispenser in seinem Sichtfeld hatte, ging er hin und die Blöcke wurden aufgenommen. Die ersten Entwicklungsschritte des A* waren ebenfalls weniger erfolgreich. Die Bewegung des Agenten war das größte Problem. \\
Für das nächste Turnier sollten die Agenten eine performante Wegfindung erhalten, die unnötigen Blöcke fallen lassen und Aufgaben klüger auswählen.

%Agent läuft mittels Spiralweg durch die Gegend \newline
%Beim erkennen eines Dispensers geht er hin \newline
%Taskverarbeitung immernoch erster Task \newline
%A* haben wir entwickelt, nur war dieser noch nicht so zuverlässig \newline
%Freischlagen des Agenten wurde eingebaut
%
%Problem: Immernoch sehr inperformante Bewegung \newline
%Hatten noch Blöcke übrig von alten Tasks \newline
%
%Verbesserung: Clevere Wegfindung \newline
%Taskverarbeitung entwickeln, damit wir besser die Tasks auswählen können \newline
%Blöcke fallen lassen, wenn wir den Block nicht brauchen \newline

\subsubsection{Turnier 3}
Im Turnier 3 konnte der Agent Aufgaben nach Rentabilität auswählen. Das herausfinden von Dispensern und Zonen wurde mittels der Aktion \emph{survey} umgesetzt. Der Agent konnte sich den Weg durch die Blöcke schlagen und den Block in die korrekte Position drehen. In das Abgeben von Tasks wurde vorher viel Zeit investiert. Durch technische Probleme am Turniertag wurden leider wieder keine Punkte gemacht. Es war ziemlich enttäuschend für die Gruppe.\\
Fürs vierte Turnier soll es endlich Punkte regnen.

%Taskverarbeitung für Block = 1 umgesetzt \newline
%Wegfindung mit survey cleverer gelöst \newline
%NextTaskPlanner implementiert, der uns den besten Plan für den besten Task hergibt \newline
%Agent kann sich durch die Gegend boxen, macht Blöcke kaputt und rotiert die Blöcke vernünftig \newline
%
%Probleme: 
%- Vorher viel dafür gesetzt, dass der Agent die Tasks abgeben kann \newline
%- Am Turniertag dann leider technische Probleme \newline
%- Sehr Enttäuschend \newline
%
%Verbesserung: WIR WOLLEN ENDLICH PUNKTEN

\subsubsection{Turnier 4}
Im vierten Turnier wurde kurz vor dem Turnier ein Fehler entdeckt, der durch die Testkonfiguration nicht aufgefallen ist. Deswegen wurde das Rollenkonzept noch kurzfristig geändert. In dem Turnier konnten dann erfolgreich die ersten Aufgaben abgegeben werden und verschiedene Rollen wurden angenommen. Die Wegfindung erfolgte effizient und die rentabelsten Aufgaben wurden ausgewählt. Allerdings gab es noch Probleme, wenn viele Agenten zusammen sind. Dies wurde vor allem deutlich, wenn sich auch die Agenten des anderen Teams um einen Dispenser oder in einer Goalzone bewegen.

\subsubsection{Turnier 5}
Beim Vorletzten Turnier hat die Gruppe schon an der Entwicklung der 2er Tasks gearbeitet. Da die Turniere vorher jedoch so schlecht liefen, hat sich die Gruppe entschieden, den Agenten mit 1er Tasks zu starten. Durch die veränderte Konfiguration sind wenige 1er Tasks entstanden und die Agenten haben leider wenige Punkte gemacht. \\
Fürs letzte Turnier sollten die 2er Tasks unbedingt funktionierten und die Verbindung von zwei Agenten.

\subsubsection{Turnier 6}
Um beim Abschlussturnier nicht letzter zu werden, wurden vorher noch einige Stunden investiert und es hat sich gelohnt. Die Entwicklung der 2er Tasks war erfolgreich, die Agenten konnten miteinander kommunizieren und sich miteinander verbinden. Die Umschaltung zwischen 1er und 2er Tasks hat sehr gut funktioniert und die Agenten haben sich auch noch selten gegenseitig blockiert. \\
Alles in allem war es das beste Turnier des Teams und zur großen Erleichterung der Teammitglieder haben die Agenten einiges an Punkten geholt.
Außerdem wurde noch eine Strategie gegen Team 1 ausgetestet. Es wurde probiert die Blöcke der Agenten des anderen Teams gezielt zu zerstören. Leider hat sich das Turnier so entwickelt, dass die Teams sich in unterschiedlichen Goalzones aufgehalten haben und unsere Teststrategie deswegen nicht aufgegangen ist.

\subsection{Problemerkennung}
Im laufe der Turniere sind uns verschiedene Probleme begegnet. Die Aufgaben, die im Hintergrund ablaufen und weniger Reaktion auf spontane Ereignisse erfordern konnten erfolgreich umgesetzt werden. Vor allem der Umgang nicht sich verändernden Bedingungen war schwierig zu implementieren. Unter anderem konnten wir im letzten Turnier feststellen, dass unsere Strategie feste Kleingruppen zu bilden, um Aufgaben mit zwei Blöcken abzugeben, nicht die effizienteste Strategie war, da wir teilweise lange Wartezeiten in Kauf nehmen mussten.

\subsection{Verbesserungsmöglichkeiten}
Eine Änderung der Strategie hätte die Wartezeiten verringern können. Die Gruppe hätte die wichtigsten Blöcke berechnen und Agenten zuteilen können, die Entscheidung, welche Agenten sich dann miteinander verbinden und eine Aufgabe abzugeben, hätten die Agenten dann unter sich treffen können.

\subsection{Lösungsstrategien}
Optimierung der Lösungsstrategien

\subsection{Interaktion mit/gegen andere Agenten}
Optimierung der Strategie zur Handhabung gegnerischer Agenten


%% Fazit
\newpage
\section{Fazit und Ausblick}

Die Gruppe wurde während des ganzen Praktikums vor immer neue Herausforderungen gestellt, die es gemeinsam zu Lösen galt. Die Einarbeitung in ein neues, unbekanntes Thema, die Programmiersprache und das Zusammenarbeiten im Team über eine größere Distanz bzw. nur Online war zu Beginn für alle neu. Durch eine offene Kommunikation hat jeder etwas beitragen können, jedes Gruppenmitglied konnte seine Erfahrungen aus dem eigenen Bereich einbringen, so dass die Gruppe erfolgreich an den Turnieren teilgenommen hat. \newline

Während der ganzen Zeit traten verschiedene Probleme auf, welche kontinuierlich diskutiert, bewertet und bearbeitet wurden. Die wöchentlichen Treffen haben zu einem guten Austausch geführt und jeden einzelnen in der Entwicklung weitergebracht. Aufgrund einiger erfolgloser Turniere musste das Team lernen, auch mit einer Niederlage oder Enttäuschung umzugehen. \newline

Die \Agents haben noch Potential, welches wir aus Zeitgründen nicht ausschöpfen konnten. Z.B. haben wir für serverseitige \textit{Events} noch keine Reaktion implementiert. Das Bilden von Kleingruppen mit den ersten beiden Agenten, die sich über den Weg laufen, ist ausbaufähig. Die \Agents haben zu oft aufeinander gewartet. Hierfür wäre eine dynamische Zusammenarbeit der \Agents zielführender  gewesen. Die anderen Agenten wurden primär als Fremdkörper interpretiert, denen ausgewichen werden konnte, jedoch wurde keine aktive Interaktion mit diesen angestrebt. Desweiteren könnte die Kommunikation innerhalb der Gruppe für bessere Koordination weiter ausgebaut werden. 



%% FAQ
%\chapter{FAQ}

Hier stehen wir Frage und Antwort :)
%
% ---- Bibliography ----
%
% BibTeX users should specify bibliography style 'splncs04'.
% References will then be sorted and formatted in the correct style.
%
% \bibliographystyle{splncs04}
% \bibliography{mybibliography}
%

\begin{thebibliography}{8}
	\bibitem{eclipse} https://www.eclipse.org/
	\bibitem{intellij} https://www.jetbrains.com/idea/
	\bibitem{netbeans} https://netbeans.apache.org/
	
	
	
	%Beispiele Am ENde zum löschen
	%\bibitem{ref_lncs1}
	%Author, F., Author, S.: Title of a proceedings paper. In: Editor,
	%F., Editor, S. (eds.) CONFERENCE 2016, LNCS, vol. 9999, pp. 1--13.
	%Springer, Heidelberg (2016). \doi{10.10007/1234567890}
	%
	%\bibitem{ref_book1}
	%Author, F., Author, S., Author, T.: Book title. 2nd edn. Publisher,
	%Location (1999)
	%
	%\bibitem{ref_proc1}
	%Author, A.-B.: Contribution title. In: 9th International Proceedings
	%on Proceedings, pp. 1--2. Publisher, Location (2010)
	%
	%\bibitem{ref_url1}
	%LNCS Homepage, \url{http://www.springer.com/lncs}. Last accessed 4
	%Oct 2017
\end{thebibliography}

\end{document}
