\section{Einleitung}

In diesem Kapitel wird die Auswahl der technischen Rahmenbedingungen, mit denen die Gruppe gearbeitet hat, genauer erläutert. Zudem erfolgt eine kurze Beschreibung der die Aufgaben, die von den Teammitgliedern bearbeitet wurden.

\subsection{Auswahl technischer Rahmenbedingungen}
Da die Schnittstelle der Programmierkenntnisse aller Gruppenmitglieder auf Java fiel und dies auch mit der MASSim Umgebung sowie der Vorlage des Agenten korrelierte, wurde in dieser Sprache entwickelt. Unsere Agentenvariante lief unter der Bezeichnung \textit{NextAgent}.  Alle Klassendateien beginnen mit dem Präfix \textit{Next-}, um die Zuordnung im Sourcecode zu erleichtern. Debugging erfolgte je nach Präferenz des Teilnehmers sowohl über Printausgabe in Echtzeit, als auch über Logfiles und Debugger. In kritischen Bereichen wurden Unittests umgesetzt. \\

Innerhalb des Quellcodes wurde auf Wunsch eines Teammitglieds eine spezielle Formatierung eingeführt. Dabei wurden die \textit{public} Methoden großgeschrieben, während \textit{private} Methoden weiterhin klein geschrieben wurden. Somit können öffentliche Methoden schneller erkannt und verwendet werden.

\subsection{Aufteilung Gruppenmitglieder - Arbeit}

Mit dem Ziel, den erarbeiteten Stand und weiteres Vorgehen zu besprechen, wurden wöchentliche Treffen abgehalten. Dieses konnten sehr konsequent umgesetzt werden. Initial wurde innerhalb der Gruppe versucht, in agiler Form mit Hilfe von \textit{issues} in GitHub zu arbeiten. Aufgrund der Komplexität des Themas und mangelnder Vorerfahrung der Teammitglieder, insbesondere im Kontext der Multiagentensysteme, haben wir mehr Zeit für die Einarbeitung des tieferen Verständnisses benötigt. 
\newpage
Aus diesem Grund wurden unser Konzept in vier Themenbereiche aufgeteilt, die wie folgt zugewiesen wurden:

\begin{itemize}
    \item \textbf{Herr Jendrny:} Festlegen der Aufgaben für den Agenten, basierend auf \textit{Tasks}, Normen und dem aktuellen Zustand der Welt.  
    \item \textbf{Frau Wolf:} Reaktion des Agenten auf die Umgebung und Wahl der auszuführenden \textit{Action}, mit Umsetzung der reaktiven Anteile.
    \item \textbf{Herr Lorenz:} Verarbeitung der \Percepts und Interaktion mit dem Server, optimale Wegfindung und Gruppenbildung, sowie Kommunikation.
    \item \textbf{Herr Loder:} Verarbeitung der Karte und das Zusammenführen der Karten für die Gruppe.
\end{itemize}

Zusätzlich gab es spontan koordinierte Treffen in kleinen Teams, um Schnittstellen zu diskutieren und Lösungen für Teilbereiche zu erarbeiten. 

\subsubsection{Alexander Lorenz} ~\\
Herr Lorenz setzte die Grundvariante des Agenten um, welche die Basis für weitere Entwicklung bildete. Der Schwerpunkt lag auf der sauberen Verarbeitung der Serverdaten, der Möglichkeit zwischen den Simulationen zu wechseln und der Fähigkeit, die ersten \textit{actions} an den Server zu senden. Dabei reagierte der Agent noch rein reaktiv und wählte die nächste Aktion basierend auf einer vorgegebenen Priorisierung. \newline

Im Verlauf des Fachpraktikums setzte Herr Lorenz die Logik der Gruppenbildung um, sowie eine Variante der Kommunikation innerhalb der Gruppe. Des Weiteren beschäftigte er sich mit den Möglichkeiten zur Optimierung der Wegfindungsalgorithmen und erweiterte die initiale A* Variante, um weitere Modifikationen, die in Kapitel \textit{\ref{kap:wegfindung}} genauer beschrieben werden. Als Schnittstelle zwischen Percepts, Karte und Entscheidungsfindung wurde viel Zeit in das Bugfixing, Fehlersuche, sowie die Interpretation des Agentenverhaltens investiert. \\

Bei Herrn Lorenz lag ein Fehler in der Konfiguration der IDE, so dass Github Commits nicht sauber zugeordnet waren. Dies wurde Anfang Juni bemerkt, und behoben.  

\subsubsection{Miriam Wolf} ~\\
Um einen Schritt nach vorn zu gehen oder einen Block zu zerstören, müssen die Agenten eine \textit{Action} an den Server schicken. Um einen optimalen Weg durch die Blöcke zu finden und sich zur Endzone durchzukämpfen, muss die nächstmögliche Aktion herausgefunden werden. Frau Wolf hat sich um die Möglichkeit gekümmert, welche Aktion die nächstmögliche ist. Sie hatte die Aufgabe, das Verhalten der Agenten in der lokalen Sicht zu bearbeiten. Die Erklärung der lokalen Sicht des Agenten wird später in Kapitel \ref{kap:lokaleSicht} genauer erläutert. \\

Weiter hatte sie die Aufgabe der Praktikumsorganisatorin inne. Hierfür musste sie die Treffen der Gruppenleiter koordinieren, die Turnierplanung übernehmen und auch die Turniere begleiten. Dazu gehörten unter anderem das Erstellen der Turnierkonfiguration oder das Starten des Servers selbst beim Turnier. Abschließend kam noch die Planung des Präsentationsnachmittags dazu und das Ausarbeiten des allgemeinen Dokumentationsteils.

\subsubsection{Sebastian Loder} ~\\
Herr Loder hat die Implementierung der Karte übernommen, welche im Detail in Kapitel \textit{\ref{erkundungDerKarte}} beschrieben ist. Die Agenten besitzen nur eine lokale Sicht, welche je \Percept abgerufen werden kann (siehe auch Kapitel \textit{\ref{kap:lokaleSicht}}). Die Informationen werden in jedem \Step übermittelt, jedoch nicht gespeichert, sodass ein Agent zunächst einmal kein Gedächtnis hat. Um zu einem späteren Zeitpunkt wieder zu bereits vorher erkundeten Bereichen zielgerichtet zurückkehren zu können, wurde die Karte implementiert, welche je Zeitschritt die lokale Sicht speichert und so die Umgebung aufnimmt. Ein wichtiger Bestandteil der Aufgabe bestand auch darin, die aufgebauten Karten der einzelnen Agenten korrekt zusammenzuführen, wenn diese sich zu einer Gruppe verbinden, siehe auch Kapitel \textit{\ref{kap:Gruppenbildung}}. \textit{Testing} und \textit{Debugging} haben in Zusammenhang mit der Karte einen relevanten Teil der Umsetzung in Anspruch genommen.

\subsubsection{Steffen Jendrny} ~\\
Herr Jendrny hat sich um die Aufgabenplanung und -verteilung unter den Agenten gekümmert. Zu Beginn der Entwicklung hat jeder Agent selbstständig entschieden, welche Aufgabe ausgewählt wird und welche Teilaufgabe momentan erfüllt werden muss. Mit der Einführung von Agentengruppen wurde die Aufgabenauswahl auf die Gruppe ausgelagert und ein Agent hat lediglich entschieden, welche Teilaufgabe momentan erfüllt werden muss. Außerdem hat Herr Jendrny zusammen mit Frau Wolf die Abgabe von Aufgaben mit mehreren Blöcken implementiert. \\
Zusammen mit Frau Wolf war Herr Jendrny bei den Lead-Treffen dabei, um Gruppe 5 zu vertreten.