\section{Einleitung}

\subsection{Auswahl technischer Rahmenbedingungen}
Für die Auswahl der technischen Rahmenbedingungen hat sich die Gruppe an der Serverprogrammierung orientiert. Der Server ist in Java implementiert und deswegen sind die Agenten ebenfalls in Java entwickelt worden. Ein weiterer Grund dafür war, dass die Schnittstelle der Programmierkenntnisse aller Gruppenmitglieder auf Java fiel. Zum Einsatz kamen verschiedene Editoren wie Eclipse\cite{eclipse}, IntelliJ\cite{intellij} oder NetBeans\cite{netbeans}.

\subsection{Aufteilung Gruppenmitglieder - Arbeit}
\subsubsection{Alexander Lorenz} ~\\
\subsubsection{Miriam Wolf} ~\\
Um einen Schritt nach vorn zu gehen oder einen Block zu zerstören, müssen die Agenten eine Aktion an den Server schicken. Um einen optimalen Weg durch die Blöcke zu finden und sich zur Endzone durchzukämpfen, muss die nächstmögliche Aktion herausgefunden werden. Frau Wolf hat sich um die Möglichkeit gekümmert, welche Aktion die nächstmögliche ist. Sie hatte die Aufgabe, das Verhalten der Agenten in der lokalen Sicht zu bearbeiten. Die Erklärung, was die lokale Sicht des Agenten ist wird später genauer erläutert. \\

Weiter hatte sie die Aufgabe der Praktikumsorganisatorin inne. Hierfür musste sie die Treffen der Gruppenleiter koordinieren, die Turnierplanung übernehmen und auch die Turniere begleiten. Dazu gehörten unter anderem das Erstellen der Turnierkonfiguration oder das Starten des Servers selbst beim Turnier. Abschließend kam noch die Planung des Präsentationsnachmittags dazu und das Ausarbeiten des allgemeinen Dokumentationsteils.

\subsubsection{Sebastian Loder} ~\\
\subsubsection{Steffen Jendrny} ~\\
Herr Jendrny hat sich um die Aufgabenplanung und -verteilung unter den Agenten gekümmert. Zu Beginn der Entwicklung hat jeder Agent selbstständig entschieden, welche Aufgabe ausgewählt wird und welche Teilaufgabe momentan erfüllt werden muss. Mit der Einführung von Agentengruppen, wurde die Aufgabenauswahl auf die Gruppe ausgelagert und ein Agent hat lediglich entschieden, welche Teilaufgabe momentan erfüllt werden muss. Außerdem hat Herr Jendrny zusammen mit Frau Wolf die Abgabe von Aufgaben mit mehreren Blöcken implementiert. \\
Zusammen mit Frau Wolf war Herr Jendrny bei den Lead-Treffen, um Gruppe 5 zu vertreten.