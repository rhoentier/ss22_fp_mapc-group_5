\section{Einleitung}

\subsection{Auswahl technischer Rahmenbedingungen}
Für die Auswahl der technischen Rahmenbedingungen hat sich die Gruppe an der Serverprogrammierung orientiert. Diese ist in Java entwickelt und desswegen sind die Agenten ebenfalls in Java entwickelt worden. Ein weiterer Grund dafür war, dass die Schnittstelle der Programmierkenntnisse aller Gruppenmitglieder auf Java fiel. Zum Einsatz kamen verschiedene Editoren wie Eclipse\cite{eclipse}, IntelliJ\cite{intellij} oder NetBeans\cite{netbeans}.

\subsection{Aufteilung Gruppenmitglieder - Arbeit}
\subsubsection{Alexander} ~\\
\subsubsection{Miriam} ~\\
Um einen Schritt nach vorn zu gehen oder einen Block zu zerstören, müssen die Agenten eine Aktion an den Server schicken. Um einen optimalen Weg durch die Blöcke zu finden und sich zur Endzone durchzukämpfen, muss die nächstmögliche Aktion herausgefunden werden. Miriam hat sich um die Möglichkeit gekümmert, welche Aktion die nächstmögliche ist. Sie hatte die Aufgabe, das Verhalten der Agenten in der lokalen Sicht zu bearbeiten. Die Erklärung, was die lokale Sicht des Agenten ist wird später genauer erläutert. \\

Weiter hatte sie die Aufgabe der Praktikumsorganisatorin inne. Hierfür musste sie die Treffen der Gruppenleiter koordinieren, die Turnierplanung übernehmen und auch die Turniere begleiten. Dazu gehörten unter anderem das Erstellen der Turnierkonfiguration oder das Starten des Servers selbst beim Turnier. Abschließend kam noch die Planung des Präsentationsnachmittags dazu und das Ausarbeiten des allgemeinen Dokumentationsteils.

\subsubsection{Sebastian} ~\\
\subsubsection{Steffen} ~\\