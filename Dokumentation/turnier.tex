\section{Turniere}

\subsubsection{Turnier 1}
Das erste Turnier wurde nach einer ziemlich kurzen Entwicklungszeit abgehalten. Um die Erfahrung eines Turniers aber nicht zu verlieren, nahm das Team dennoch teil. Zu diesem Zeitpunkt waren die Agenten gerade in der Lage, die Karte zu erkunden. Der Fokus der ersten Entwicklungszeit lag auf der Erkundung der Karte, der Kommunikation mit dem Server und sich mit den Aufgaben und den Gegebenheiten des Turniers vertraut zu machen. Aufgaben konnten zu dem Zeitpunkt nicht abgegeben werden.\\
Das Ziel für das nächste Turnier war es, die Blöcke zu zerstören und zu den verschiedenen Zonen und Dispensern zu gehen. Eine bessere Wegfindung war ebenfalls notwendig.

%%Tasks mit Blöcken = 1 \newline
%Erster Task der liste \newline
%Agenten liefen mittels Random-Weg durch die Gegend. \newline
%Beim Entdecken vom Dispenser sind sie hingelaufen und haben sich einen Block genommen \newline
%Weiter random durch die Gegend bis zur Endzone \newline
%
%Probleme hier: Der Agent hat sich den Weg noch nicht freigehauen, somit hatten wir keine gute Methode zur Endzone zu kommen \newline
%Verbesserung: Bessere Wegfindung und wege freihauen

\subsubsection{Turnier 2}
Beim zweiten Turnier wurde die Kartierung verändert, warn allerdings immernoch simpel. Der Agent ist die Karte mit einem spiralförmigen Weg abgelaufen und hat somit verschiedene Dinge entdeckt. Sobald er einen Dispenser in seinem Sichtfeld hatte, ging er hin und die Blöcke wurden aufgenommen. Die ersten Entwicklungsschritte des A* waren ebenfalls weniger erfolgreich. Die Bewegung des Agenten war das größte Problem. \\
Für das nächste Turnier sollten die Agenten eine performante Wegfindung erhalten, die unnötigen Blöcke fallen lassen und Aufgaben klüger auswählen.

%Agent läuft mittels Spiralweg durch die Gegend \newline
%Beim erkennen eines Dispensers geht er hin \newline
%Taskverarbeitung immernoch erster Task \newline
%A* haben wir entwickelt, nur war dieser noch nicht so zuverlässig \newline
%Freischlagen des Agenten wurde eingebaut
%
%Problem: Immernoch sehr inperformante Bewegung \newline
%Hatten noch Blöcke übrig von alten Tasks \newline
%
%Verbesserung: Clevere Wegfindung \newline
%Taskverarbeitung entwickeln, damit wir besser die Tasks auswählen können \newline
%Blöcke fallen lassen, wenn wir den Block nicht brauchen \newline

\subsubsection{Turnier 3}
Im Turnier 3 konnte der Agent Aufgaben nach Rentabilität auswählen. Das herausfinden von Dispensern und Zonen wurde mittels der Aktion \emph{survey} umgesetzt. Der Agent konnte sich den Weg durch die Blöcke schlagen und den Block in die korrekte Position drehen. In das Abgeben von Tasks wurde vorher viel Zeit investiert. Durch technische Probleme am Turniertag wurden leider wieder keine Punkte gemacht. Es war ziemlich enttäuschend für die Gruppe.\\
Fürs vierte Turnier soll es endlich Punkte regnen.

%Taskverarbeitung für Block = 1 umgesetzt \newline
%Wegfindung mit survey cleverer gelöst \newline
%NextTaskPlanner implementiert, der uns den besten Plan für den besten Task hergibt \newline
%Agent kann sich durch die Gegend boxen, macht Blöcke kaputt und rotiert die Blöcke vernünftig \newline
%
%Probleme: 
%- Vorher viel dafür gesetzt, dass der Agent die Tasks abgeben kann \newline
%- Am Turniertag dann leider technische Probleme \newline
%- Sehr Enttäuschend \newline
%
%Verbesserung: WIR WOLLEN ENDLICH PUNKTEN

\subsubsection{Turnier 4}
Im vierten Turnier wurde kurz vor dem Turnier ein Fehler entdeckt, der durch die Testkonfiguration nicht aufgefallen ist. Deswegen wurde das Rollenkonzept noch kurzfristig geändert. In dem Turnier konnten dann erfolgreich die ersten Aufgaben abgegeben werden und verschiedene Rollen wurden angenommen. Die Wegfindung erfolgte effizient und die rentabelsten Aufgaben wurden ausgewählt. Allerdings gab es noch Probleme, wenn viele Agenten zusammen sind. Dies wurde vor allem deutlich, wenn sich auch die Agenten des anderen Teams um einen Dispenser oder in einer Goalzone bewegen.

\subsubsection{Turnier 5}
Beim Vorletzten Turnier hat die Gruppe schon an der Entwicklung der 2er Tasks gearbeitet. Da die Turniere vorher jedoch so schlecht liefen, hat sich die Gruppe entschieden, den Agenten mit 1er Tasks zu starten. Durch die veränderte Konfiguration sind wenige 1er Tasks entstanden und die Agenten haben leider wenige Punkte gemacht. \\
Fürs letzte Turnier sollten die 2er Tasks unbedingt funktionierten und die Verbindung von zwei Agenten.

\subsubsection{Turnier 6}
Um beim Abschlussturnier nicht letzter zu werden, wurden vorher noch einige Stunden investiert und es hat sich gelohnt. Die Entwicklung der 2er Tasks war erfolgreich, die Agenten konnten miteinander kommunizieren und sich miteinander verbinden. Die Umschaltung zwischen 1er und 2er Tasks hat sehr gut funktioniert und die Agenten haben sich auch noch selten gegenseitig blockiert. \\
Alles in allem war es das beste Turnier des Teams und zur großen Erleichterung der Teammitglieder haben die Agenten einiges an Punkten geholt.
Außerdem wurde noch eine Strategie gegen Team 1 ausgetestet. Es wurde probiert die Blöcke der Agenten des anderen Teams gezielt zu zerstören. Leider hat sich das Turnier so entwickelt, dass die Teams sich in unterschiedlichen Goalzones aufgehalten haben und unsere Teststrategie deswegen nicht aufgegangen ist.

\subsection{Problemerkennung}
Im laufe der Turniere sind uns verschiedene Probleme begegnet. Die Aufgaben, die im Hintergrund ablaufen und weniger Reaktion auf spontane Ereignisse erfordern konnten erfolgreich umgesetzt werden. Vor allem der Umgang nicht sich verändernden Bedingungen war schwierig zu implementieren. Unter anderem konnten wir im letzten Turnier feststellen, dass unsere Strategie feste Kleingruppen zu bilden, um Aufgaben mit zwei Blöcken abzugeben, nicht die effizienteste Strategie war, da wir teilweise lange Wartezeiten in Kauf nehmen mussten.

\subsection{Verbesserungsmöglichkeiten}
Eine Änderung der Strategie hätte die Wartezeiten verringern können. Die Gruppe hätte die wichtigsten Blöcke berechnen und Agenten zuteilen können, die Entscheidung, welche Agenten sich dann miteinander verbinden und eine Aufgabe abzugeben, hätten die Agenten dann unter sich treffen können.

\subsection{Lösungsstrategien}
Optimierung der Lösungsstrategien

\subsection{Interaktion mit/gegen andere Agenten}
Optimierung der Strategie zur Handhabung gegnerischer Agenten
