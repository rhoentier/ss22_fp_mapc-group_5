\section{Turniere}
In diesem Kapitel wird auf die Turniere eingegangen. Zunächst werden die Schwierigkeiten im Turnierverlauf erklärt und anschließend passende Lösungsstrategien. Im letzten Kapitel wird kurz auf die Interaktion mit und gegen andere Agenten beschrieben.

\subsection{Schwierigkeiten im Turnierverlauf}
Zu Beginn der Turniersaison steckten die \Agents noch in den Kinderschuhen. Sie mussten sich zunächst auf der Karte orientieren und das gesehene Verarbeiten. Die Kartierung war zu Beginn noch ein zufälliger Weg, der in weiterer Folge der Turniere auf einen Spiralförmigen Weg bis hin zu dem richtigen Wegfindungsalgorithmus A* ausgearbeitet wurde. Die Kommunikation mit dem Server musste erst umgesetzt und verstanden werden ebenso wie die abzugebenden Aufgaben. Die Probleme zu Beginn lagen darin, nicht unnötig gegen Wände zu laufen oder sogar die Blöcke zu zerstören. \\

In weiterer Folge der Turniere wurden die Hindernisse zerstört und die \Agents sind gezielt zu den Dispensern gelaufen, sobald sie welche gesehen haben. Die ersten Versuche mit dem A* waren weniger erfolgreich und stellte das Team vor eine größere Herausforderung, die es zu Lösen gab. Ein weiteres Problem bestand darin, dass die Agenten zu lange gebraucht haben, um die \Dispenser oder die \GoalZones / \RoleZones zu finden. \\

Bei Halbzeit des Praktikums haben die \Agents einen großen Schritt nach vorn machen können. Sie konnten sich den Weg durch die \Obstacles schlagen und die benötigten Blöcke in die korrekte Position drehen. Sie konnten mittels einer Aktion \textit{survey} schneller zu dem gesuchten Ziel gehen, was das Team vor ein weiteres Problem gestellt hat. \\

\begin{figure}
	\begin{minipage}{0.5\textwidth}
	Bei der Karte setzten wir vom Beginn der Entwicklung auf eine randlose Karte. Dies führte zu einigen Problemen, die auch unsere Performance in den Turnieren stark beeinträchtigten. Z.B. trat ein Fehler im Verhalten zu einem bestimmten Zeitpunkt auf, welcher zuerst als gewolltes Feature fehlinterpretiert wurde. Die \Agents wanderten nach links oben und haben dadurch die Größe der Karte massiv gestreckt, wie in Abbildung \ref{fig:kartenwiederholung} dargestellt. Dies führte zu sehr vielen redundanten Bewegungen und kostete leider unnötig Zeit.
	\end{minipage}
	\begin{minipage}{0.4\textwidth}
	\includegraphics[width=170px]{bilder/karte.png}
	\caption{Beispiel für eine gestreckte Karte basierend auf einer 24x24 Welt }
	\label{fig:kartenwiederholung}
	\end{minipage}
\end{figure}

In unseren Versuchen setzten wir auf spezialisierte Umgebungen, um Teilbereiche zu optimieren. Hier trat das Verhalten nur selten auf und wurde dadurch kaum beobachtet. In der Turnierumgebung wurde es jedoch sehr dominant, und spätestens im fünften Turnier wurde dieses Verhalten als extrem kritisch eingestuft und konnte nach einiger Zeit beseitigt werden. Bis zum vierten Turnier hatten wir noch Probleme mit der Synchronisierung der Karte, welche durch dieses Verhalten verschärft wurden.
Das nächste Problem, welches das Team zu bewältigen hatte war die Auswahl der Tasks nach Profitabilität und die Gruppenbildung, damit die \Agents im Team die Aufgaben lösen können. 

Im letzten Drittel des Praktikums konnten die \Agents Aufgaben auswählen und in einer Gruppe zusammenarbeiten. Hierbei ist aufgefallen, dass sich die \Agents gegenseitig im Weg standen und somit sich selbst blockiert haben. Dieses Problem hat eines der Gruppenmitglieder extrem lange aufgehalten. \newline

Beim Abschlussturnier haben die \Agents Aufgaben profitabel ausgewählt, in Gruppen zusammengearbeitet, die benötigten \Dispenser oder Zonen erfolgreich gesucht und sind mit einem performanten Weg zu den Zielen gelaufen. Die gegenseitige Blockierung konnte mittels Kommunikation gelöst werden, sodass am Ende viele Punkte erzielt werden konnten. \newline

\subsection{Lösungsstrategien}
Die Lösungsstrategien werden zu den Themenbereichen aufgeteilt und kurz erläutert:

\subsubsection{Wegfindung}
Zu Beginn sind die \Agents zufällig über die Karte gelaufen. Als einfachen Wegfindung wurde dann ein Spiralförmiger Weg gewählt, um die Karte zu erkunden und zufällig passende \Dispenser zu finden. Die Entwicklung des A* begann ziemlich früh und wurde die meiste Zeit des Praktikums optimiert und weiterentwickelt. 

\subsubsection{Kommunikation}
Die Kommunikation unter den \Agents wurde bei uns nur in der Gruppe umgesetzt. Da die Agenten meistens in der Mitte des Spiels zu einer Gruppe gefunden haben, hat es uns nicht eingeschränkt. Hierfür wurde eine Art Nachrichtenbox erstellt, aus der die \Agents Nachrichten auslesen oder hineinlegen konnten. Auch das Problem mit dem gegenseitigen Blockieren konnte mittels der Kommunikation gelöst werden

\subsubsection{Karte}
Während des Suchens der verschiedenen Ziele wurde unsere Karte immer nach links oben erweitert. Dies war ein simpler Fehler, der die Gruppe einige Wochen gekostet hat. Dabei hat sich beim zufälligen Suchen ein Fehler eingeschlichen, der die Agenten immer in eine Richtung hat gehen lassen. Nachdem der Fehler behoben war, blieb die Karte kleiner und der Wegfindealgorithmus hat noch besser funktioniert.


\subsection{Interaktion mit/gegen andere Agenten}
Die anderen Agenten wurden primär als Fremdkörper interpretiert, denen ausgewichen werden konnte, jedoch wurde keine aktive Interaktion mit diesen angestrebt. Ein Ansatz, statt den Agenten die Blöcke anzugreifen, wurde durch das MASSim Regelwerk verhindert. 