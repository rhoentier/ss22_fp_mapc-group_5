% This is samplepaper.tex, a sample chapter demonstrating the
% LLNCS macro package for Springer Computer Science proceedings;
% Version 2.21 of 2022/01/12
%
\documentclass[runningheads]{llncs}
\pagenumbering {gobble}
%
\usepackage[T1]{fontenc}
\usepackage{hyperref}
\usepackage{babel}
% T1 fonts will be used to generate the final print and online PDFs,
% so please use T1 fonts in your manuscript whenever possible.
% Other font encondings may result in incorrect characters.
%
\usepackage{graphicx}
% Used for displaying a sample figure. If possible, figure files should
% be included in EPS format.
%
% If you use the hyperref package, please uncomment the following two lines
% to display URLs in blue roman font according to Springer's eBook style:
%\usepackage{color}
%\renewcommand\UrlFont{\color{blue}\rmfamily}
%
\begin{document}
%
\title{FAQ des NextAgent}
\subtitle{Gruppe 5}
%
%\titlerunning{Abbreviated paper title}
% If the paper title is too long for the running head, you can set
% an abbreviated paper title here
%
\author{Jan Steffen Jendrny,
Sebastian Loder,
Alexander Lorenz und
Miriam Wolf
}
%
\authorrunning{Jendrny S., Loder S., Lorenz A., Wolf M.}
% First names are abbreviated in the running head.
% If there are more than two authors, 'et al.' is used.
%
\institute{Fernuniversität Hagen, Universitätsstraße 47, 58097 Hagen
\url{https://www.fernuni-hagen.de/}}
%
\maketitle              % typeset the header of the contribution
%
%\begin{abstract}
%The abstract should briefly summarize the contents of the paper in
%150--250 words.
%
%\keywords{First keyword  \and Second keyword \and Another keyword.}
%\end{abstract}
%
%
%


Die Fragen und Antworten ist nicht als wissenschaftlicher Teil anzusehen.

%% FAQ
\chapter{FAQ}

Hier stehen wir Frage und Antwort :)
%
% ---- Bibliography ----
%
% BibTeX users should specify bibliography style 'splncs04'.
% References will then be sorted and formatted in the correct style.
%
% \bibliographystyle{splncs04}
% \bibliography{mybibliography}
%

\end{document}
